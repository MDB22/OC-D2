\subsection{Locating Joe}
Upon successfully reaching the end of the transit corridor, the aircraft will reduce altitude to 150 ft and slow to a controllable fixed wing flight, while maintaining speed above stall speed. Inside the Remote Landing Site a spiral search pattern will be implemented. It is assumed that the Remote Landing Site is centered at Joe's location and his location has an accuracy of plus or minus 100 meters.\\

With this in mind, the drone will navigate itself to the border of the reported area (lining up with a circle whose radius is 100 meters from Joe’s reported location). Once the aircraft has lined up with the circle it will start processing the images received from the camera mounted in the nose of the aircraft. It will continue along this path, every revolution bringing it slightly closer to the center until the 200 meter diameter circle has been searched in it’s entirety.\\

An on board companion computer (Raspberry Pi™) will use open source visual computation software OpenCV to process the incoming images in an attempt to identify any object on the ground that might match Joe's description. The visual processing will include (blue) color masking and recognition, edge detection and possibly human/facial recognition, as well as a machine learning layer. Objects will be given scores based on the likelihood that they are Joe, and when an object scores above a threshold the aircraft will attempt to further identify the object. At this point the approximate position of the object and the current position in the holding pattern will be recorded.\\

The aircraft will then move from fixed wing mode to hover-mode.Once in hover mode the aircraft will return to the recorded position and reacquire the object. Using the mounting angle of the camera and repeated altitude measurements (checked with GPS and barometric sensors) the Raspberry Pi will calculate the distance from the aircraft to the object. Using these more precise conditions the drone will move closer to the object and further attempt to identify it as Joe. Once the aircraft is satisfied that Joe has been found it will begin maneuvers (see below) in an attempt to perform a safe landing.\\

Should the aircraft decide that the item was a false positive, it will regain altitude, discard the object in memory, re-engage fixed wing flight and return to the stored location in the holding pattern to continue the search, repeating the above process with any object scoring a higher identification score than the threshold. 

\subsection{Finding a landing point}
Having calculated Joe's position, the area around Joe is recorded and an approximate landing site computed using OpenCV to find a  ``debris-free'' site, more than 30 meters from Joe, but less than 50 meters (for maximum points). The aircraft will fly in reverse until the camera is upon the proposed landing area and perform a more thorough search of the area with OpenCV in an attempt to identify any likelihood of failure upon landing due to collision. If any object is detected, the aircraft will strafe right while still keeping Joe centered and measure the next site. Then the aircraft will slowly move forward and land in the appropriate site, and then disarm.