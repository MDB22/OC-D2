\label{sec:risk-management}
This section addresses how the team will manage and mitigate the risks identified in Section \ref{sec:risk-assessment}. Continuing to follow the Risk Assessment template, the tables below will contain the following information in each row: the hazard (as identified previously), any control measures used to mitigate or manage the hazard, and the resultant risk rating, after implementing the control measures.\\

\begin{table}[!ht]
	\label{tab:management-electrical}
	\centering
	\begin{tabularx}{\textwidth}{|Y|Y|c|}
		\hline		
		\MULTROW{\textbf{Risk}} & \MULTROW{\textbf{Control Measures}} & \MULTROW{\textbf{Residual Risk Rating}}[70pt]\\
		& & \\
		\hline
		\MULTROW{Electrocution} & Test and tag GCS hardware, eliminate exposed wiring (heat shrink) in aircraft & \MULTROW{Low}\\
		\hline
		\MULTROW{Overcharge of batteries, causing explosion} & Li-Po charger with overcharge detection and cell management & \MULTROW{Low}\\
		\hline
		\MULTROW{Impact to batteries, causing explosion} & Transport using flame retardant, cushioned cases  & \MULTROW{Low}\\
		\hline
		\MULTROW{Insufficient power to complete mission} & Batteries of sufficient quantity and capacity to power aircraft for over 1 hour & \MULTROW{Low}\\
		\hline
		\MULTROW{Loss of power to motors} & Batteries of sufficient quantity and capacity to power aircraft for over 1 hour & \MULTROW{Low}\\		
		\hline
		\MULTROW{Loss of power to autopilot/flight controls}[*][3] & Separate power systems for autopilot/flight controls, and auxiliary systems & \MULTROW{Low}[*][3]\\
		\hline
		System, or electrical connection failure due to heat inside canopy & Ventilation and high gauge (high current) wire used inside aircraft & \MULTROW{Low}\\
		\hline
	\end{tabularx} 
	\caption{Risk Management - Electrical Hazards}
\end{table}

\begin{table}[!ht]
\label{tab:management-autonomy}
\centering
\begin{tabularx}{\textwidth}{|Y|Y|c|}
	\hline
	\MULTROW{\textbf{Risk}} & \MULTROW{\textbf{Control Measures}} & \MULTROW{\textbf{Residual Risk Rating}}[70pt]\\
	& & \\
	\hline
	Danger to personnel due to non-vertical launch & Launch aircraft a safe distance from personnel, as prescribed by the organisers & \MULTROW{Low}\\
	\hline
	\MULTROW{Non-vertical launch due to high winds} & Challenge will not proceed if the 10 minute average wind speed exceeds 25kts & \MULTROW{Low}\\
	\hline
	Non-vertical launch due to motor failure & Aircraft can be disabled from GCS & Low\\
	\hline
	\MULTROW{Loss of aircraft control while grounded} & Aircraft can be disabled from GCS, or external e-stop & \MULTROW{Low}\\
	\hline
	\MULTROW{Loss of GPS during landing maneuvers}[*][3] & Aircraft will estimate position, attitude and altitude using IMU, ultrasonic sensor and accelerometer, respectively & \MULTROW{Low}[*][3]\\
	\hline
	\MULTROW{Harm to personnel when arming aircraft} & Aircraft will wait for one minute before activating motors after arming & \MULTROW{Low}\\
	\hline
\end{tabularx} 
\caption{Risk Management - Autonomous Takeoff and Landing}
\end{table}

GPS failure in all modes, hovering without GPS

\begin{table}[!ht]
	\label{tab:management-inflight}
	\centering
	\begin{tabularx}{\textwidth}{|Y|Y|c|}
		\hline
		\MULTROW{\textbf{Risk}} & \MULTROW{\textbf{Control Measures}} & \MULTROW{\textbf{Residual Risk Rating}}[70pt]\\
		& & \\
		\hline
		Loss of aircraft control & \MULTROW{Autopilot implements failsafe termination} & \MULTROW{Low} \\
		(Autopilot failure or lock-up) & & \\
		\hline
		Loss of aircraft control (Propeller or power loss) & Failsafe is commanded by Raspberry Pi upon erratic/uncontrolled behaviour & \MULTROW{Low} \\		
		\hline
		Loss of GPS link only to GCS & Likely & Low \\
		\hline
		Loss of telemetry/radio link only to GCS & Likely & Low \\
		\hline
		Loss of both telemetry and GPS to GCS & Possible & Low \\
		\hline
		GCS failure & \MULTROW{Possible} & \MULTROW{Low}\\
		(aircraft loses communication with GCS) & & \\
		\hline
		Geofence breach & Likely & Medium \\
		\hline
		\MULTROW{Failure caused by software bug}[*][3] & All software thoroughly tested in all flight modes using software and hardware in the loop (SITL, HITL) simulation & \MULTROW{Low}[*][3]\\
		\hline
	\end{tabularx} 
	\caption{Risk Assessment - In-flight Hazards}
\end{table}