\subsection{System Diagrams}
Software and hardware architecture diagrams from final report.

\subsection{Aeronautical Requirements}
\subsubsection{Endurance and Speed Requirements}
In modeling the performance of our aircraft, we assume that the 2 takes-offs and 2 landings for the mission will average no more than 5 minutes each, including accepting the sample from Outback Joe. With 60 minutes allocated for the task, the aircraft must travel the transit corridor to Joe and back, 32 nautical miles, within 40 minutes. With this assumption, the aircraft must be capable of traveling at 48.6 knots. 

\subsubsection{Aircraft Capabilities}
\begin{itemize}
	\item[\textbf{Speed}] As mentioned in Section \ref{sec:intro}, MAS' UAV is built on the Skywalker X8 airframe. The X8 has been demonstrated by many users to achieve airspeeds in excess of 140km/h using similar battery and motor selections.
	\item[\textbf{Lift}] The stall speed of our aircraft is estimated to be 25.3 knots, according to empirical data on the Skywalker X8
	\item[\textbf{Endurance}] Battery tests have shown that 4 vertical take-off and landing maneuvers, reaching YYYYY feet, lasting 1-2 minutes each, consumes only XXXXXX\% battery. 
\end{itemize}

As the aircraft is not yet fully complete, the requirements above are estimation/calculations only. Full testing and validation will be performed prior to Deliverable 3, once the aircraft has progressed in development. At this point proper range and endurance tests to will be conducted get a better estimation of cruise speeds, take off, landing and sample pick up times, as well as maximum flight time.

\subsection{Flight Termination System}
\subsubsection{Design}

\subsubsection{State Machine}

\subsubsection{Analysis}
Effect of flight termination system (control surface positions, etc.)

Response to various failures.\\

Design, state machine and transitions.\\

Analysis of termination system capabilities.

\subsection{Geofence System}
Discuss processing of data, set up of Geofence around mission, implementation of soft Geofence.

Geofences are represented differently depending on plane or copter:
- Copter geofences are cylindrical
- Plane geofences are (up to) 18 point polygons

Approach 1 - Switching aircraft configuration mid-flight:
- Assign cylindrical geofence at base on startup
- Upon transition, assign full, polygonal geofence for transit corridor
- Upon reaching landing site and transitioning to copter, assign cylindrical geofence for landing site
- Transition, redefine transit corridor geofence and return to base
- Redefine cylindrical geofence and land at base

Approach 2 - Using only plane mode:
- Assign square polygon for base geofence
- Once in the air, assign new multi-point polygon for transit corridor (which should encompass base and landing site)
- Assign square polygon for landing sight once near GPS position
- Reassign transit corridor when back in the air
- Reassign base geofence when finish transit

We will most likely use approach 2, given what Bede has said regarding firmware implementation.

We also need to implement some procedure for implementing the ``soft geofence'', where we define a fake geofence smaller than the actual geofence, to help prevent us needing flight termination; needs research.

\subsection{Radio Equipment}
Discussion of equipment and frequencies, and compliance with CASA regulations.

RFD900 Radio Modem, 902-928MHz operating frequency, FHSS, up to 50 frequency hopping channels, compliant with FCC Part 15.247, AS/NZS 4268:2008. Making use of Item 52 of ACMA Radiocommunications (LIPD) Class Licence 2000.