\label{sec:risk-management}
\subsection{Backup Systems}
In order to protect against various failures during flight, a number of backup systems will be used on the UAV, enhancing the safety of each flight. These systems are:
\begin{enumerate}
	\item Isolated power supply for PixHawk flight controller, separate from sensor and actuator power supply
	\item Redundant position sensors to complement GPS, including IMU and accelerometer in PixHawk, altimeter (high altitude), and LiDARLite and ultrasonic sensor (low altitude) mounted underneath airframe
	\item A backup flight termination protocol programmed into the Raspberry Pi to complement automatic flight termination in the PixHawk
\end{enumerate}

\subsection{Software Development}
Give the nature of real-world autonomous systems, it is impossible to develop software that can respond to every potential situation. To make our software as robust as possible all of our software systems are tested through DroneKit's Software in the Loop (SITL) simulator, and visualised in the Mission Planner ground control station. Once the software has been validated in the simulator, we progress to Hardware in the Loop (HITL) testing, and validate the software on our UAV platform.

\subsection{Risk Control Measures}
This section addresses how the team will manage and mitigate the risks identified in Section \ref{sec:risk-assessment}. Continuing to follow the Risk Assessment template, the tables below will contain the following information in each row: the hazard (as identified previously), any control measures used to mitigate or manage the hazard, and the resultant risk rating, after implementing the control measures.\\

\begin{table}[!ht]
	\label{tab:management-electrical}
	\centering
	\begin{tabularx}{\textwidth}{|Y|Y|c|}
		\hline		
		\MULTROW{\textbf{Risk}} & \MULTROW{\textbf{Control Measures}} & \MULTROW{\textbf{Residual Risk Rating}}[70pt]\\
		& & \\
		\hline
		\MULTROW{Electrocution, up to 120A} & Test and tag GCS hardware, eliminate exposed wiring (heat shrink) in aircraft & \MULTROW{Low}\\
		\hline
		\MULTROW{Overcharge of batteries, causing explosion} & Li-Po charger with overcharge detection and cell management & \MULTROW{Low}\\
		\hline
		\MULTROW{Impact to batteries, causing explosion} & Transport using flame retardant, cushioned cases  & \MULTROW{Low}\\
		\hline
		\MULTROW{Insufficient power to complete mission} & Batteries of sufficient quantity and capacity to power aircraft for over 1 hour & \MULTROW{Low}\\
		\hline
		\MULTROW{Loss of power to motors} & Batteries of sufficient quantity and capacity to power aircraft for over 1 hour & \MULTROW{Low}\\		
		\hline
		\MULTROW{Loss of power to autopilot/flight controls}[*][3] & Separate power systems for autopilot/flight controls, and auxiliary systems & \MULTROW{Low}[*][3]\\
		\hline
		System, or electrical connection failure due to heat inside canopy & Ventilation and high gauge (high current) wire used inside aircraft & \MULTROW{Low}\\
		\hline
	\end{tabularx} 
	\caption{Risk Management - Electrical Hazards}
\end{table}

\begin{table}[!ht]
\label{tab:management-autonomy}
\centering
\begin{tabularx}{\textwidth}{|Y|Y|c|}
	\hline
	\MULTROW{\textbf{Risk}} & \MULTROW{\textbf{Control Measures}} & \MULTROW{\textbf{Residual Risk Rating}}[70pt]\\
	& & \\
	\hline
	Danger to personnel due to non-vertical launch & Launch aircraft a safe distance from personnel, as prescribed by the organisers & \MULTROW{Low}\\
	\hline
	\MULTROW{Non-vertical launch due to high winds} & Challenge will not proceed if the 10 minute average wind speed exceeds 25kts & \MULTROW{Low}\\
	\hline
	Non-vertical launch due to motor failure & Aircraft can be disabled from GCS & Low\\
	\hline
	\MULTROW{Loss of aircraft control while grounded} & Aircraft can be disabled from GCS, or external e-stop & \MULTROW{Low}\\
	\hline
	\MULTROW{Loss of GPS during landing maneuvers}[*][3] & Aircraft will estimate position, attitude and altitude using IMU, ultrasonic sensor and accelerometer, respectively & \MULTROW{Low}[*][3]\\
	\hline
	\MULTROW{Harm to personnel when arming aircraft} & Aircraft will wait for one minute before activating motors after arming & \MULTROW{Low}\\
	\hline
\end{tabularx} 
\caption{Risk Management - Autonomous Takeoff and Landing}
\end{table}

\begin{table}[!ht]
	\label{tab:management-inflight}
	\centering
	\begin{tabularx}{\textwidth}{|Y|Y|c|}
		\hline
		\MULTROW{\textbf{Risk}} & \MULTROW{\textbf{Control Measures}} & \MULTROW{\textbf{Residual Risk Rating}}[70pt]\\
		& & \\
		\hline
		Loss of aircraft control (Autopilot failure or lock-up) & Autopilot automatically executes inbuilt failsafe termination & \MULTROW{Low} \\
		\hline
		Loss of aircraft control (Propeller or power loss) & Failsafe is commanded by Raspberry Pi upon erratic/uncontrolled behaviour & \MULTROW{Low} \\		
		\hline
		\MULTROW{Loss of GPS link to UAV (Hover)}[*][4] & Switch to Pos Hold mode until a GPS fix is acquired; land aircraft if $t$ seconds have elapsed, where $t$ is min time to Geofence crossing from last position & \MULTROW{Low}[*][4] \\
		\hline
		\MULTROW{Loss of GPS link to UAV (Fixed wing)}[*][4] & Switch to Circle mode until a GPS fix is acquired; land aircraft if $t$ seconds have elapsed, where $t$ is min time to Geofence crossing from last position & \MULTROW{Low}[*][4]\\
		\hline
		\MULTROW{Loss of telemetry/radio link to GCS}[*][3] \MULTROW{(Hover)}[*][3] & If a heartbeat has not been detected for 10s, switch to PosHold mode; if no heartbeat detected for 30s, immediately land and declare mission failure & \MULTROW{Low}[*][4] \\
		\hline
		\MULTROW{Loss of telemetry/radio link to GCS}[*][3] \MULTROW{(Fixed Wing)}[*][3] & If a heartbeat has not been detected for 10s, switch to Circle mode; if no heartbeat detected for 30s, transition to Hover, and follow Hover procedures & \MULTROW{Low}[*][4] \\
		\hline
		\MULTROW{Loss of both telemetry and GPS to GCS} & Aircraft is deemed out of control, initiate flight termination & \MULTROW{Low} \\
		\hline
		GCS failure or lockup (aircraft loses communication with GCS) & Attempt to restart GCS; aircraft follows procedure for loss of radio link & \MULTROW{Low}\\
		\hline
		\MULTROW{Geofence breach} & Implement soft Geofence to initiate Geofence avoidance near boundaries & \MULTROW{Low} \\
		\hline
	\end{tabularx} 
	\caption{Risk Management - In-flight Hazards}
\end{table}

\clearpage
\subsection{Pre-Flight Checklist}
Before commencing a flight, the following checks are undertaken:
\begin{enumerate}
	\item Check integrity of airframe for damage
	\item Verify telemetry link between aircraft and GCS
	\item Verify calibration and responsiveness of accelerometer and gyroscope
	\item Verify calibration of remote control, in the event a manual override is required
	\item Test that all control surfaces are controllable and operational
	\item Verify GPS lock by checking GCS, and the state of the autopilot
	\item Verify calibration and responsiveness of compass
	\item Arm the aircraft
	\item Verify actuation of control surfaces in Stabilise mode
	\item Switch to Loiter mode and confirm successful arm and GPS fix
\end{enumerate} 

If all of these tests are passed, the aircraft is ready and safe to fly.