\subsection{System Diagram}
Software and hardware architecture diagrams from final report.

\subsection{Aeronautical Requirements}
Discuss flight capabilities? Speed, endurance, range...

\subsection{Flight Termination System}
Effect of flight termination system (control surface positions, etc.)

Response to various failures.\\

Design, state machine and transitions.\\

Analysis of termination system capabilities.

\subsection{Geofence System}
Discuss processing of data, set up of Geofence around mission, implementation of soft Geofence.

Geofences are represented differently depending on plane or copter:
- Copter geofences are cylindrical
- Plane geofences are (up to) 18 point polygons

Approach 1 - Switching aircraft configuration mid-flight:
- Assign cylindrical geofence at base on startup
- Upon transition, assign full, polygonal geofence for transit corridor
- Upon reaching landing site and transitioning to copter, assign cylindrical geofence for landing site
- Transition, redefine transit corridor geofence and return to base
- Redefine cylindrical geofence and land at base

Approach 2 - Using only plane mode:
- Assign square polygon for base geofence
- Once in the air, assign new multi-point polygon for transit corridor (which should encompass base and landing site)
- Assign square polygon for landing sight once near GPS position
- Reassign transit corridor when back in the air
- Reassign base geofence when finish transit

We will most likely use approach 2, given what Bede has said regarding firmware implementation.

We also need to implement some procedure for implementing the ``soft geofence'', where we define a fake geofence smaller than the actual geofence, to help prevent us needing flight termination; needs research.

\subsection{Radio Equipment}
Discussion of equipment and frequencies, and compliance with CASA regulations.