\label{sec:risk-management}
\subsection{Mitigating Risks identified in Section \ref{sec:risk-assessment}}
This section addresses how the team will manage and mitigate the risks identified in Section \ref{sec:risk-management}. Continuing to follow the Risk Assessment template, the tables below will contain the following information in each row: the hazard (as identified previously), any control measures used to mitigate or manage the hazard, and the resultant risk rating, after implementing the control measures.\\

\begin{table}[!ht]
	\label{tab:management-electrical}
	\centering
	\begin{tabularx}{\textwidth}{|Y|Y|c|}
		\hline		
		\MULTROW{\textbf{Risk}} & \MULTROW{\textbf{Control Measures}} & \MULTROW{\textbf{Residual Risk Rating}}[70pt]\\
		& & \\
		\hline
		\MULTROW{Electrocution} & Test and tag GCS hardware, eliminate exposed wiring (heat shrink) in aircraft & \MULTROW{Medium}\\
		\hline
		Impact to batteries, causing explosion & Transport using flame retardant  & Medium\\
		\hline
		Loss of motor power & Possible & Medium\\		
		\hline
	\end{tabularx} 
	\caption{Risk Management - Electrical Hazards}
\end{table}

\begin{table}[!ht]
\label{tab:management-autonomy}
\centering
\begin{tabularx}{\textwidth}{|Y|Y|c|}
	\hline
	\MULTROW{\textbf{Risk}} & \MULTROW{\textbf{Control Measures}} & \MULTROW{\textbf{Residual Risk Rating}}[70pt]\\
	& & \\
	\hline
	Danger to personnel due to non-vertical launch & Launch aircraft a safe distance from personnel, as prescribed by the organisers & \MULTROW{Low}\\
	\hline
	\MULTROW{Non-vertical launch due to high winds} & Challenge will not proceed if the 10 minute average wind speed exceeds 25kts & \MULTROW{Low}\\
	\hline
	Non-vertical launch due to motor failure & Aircraft can be disabled from GCS & Low\\
	\hline
	\MULTROW{Loss of aircraft control while grounded} & Aircraft can be disabled from GCS, or external e-stop & \MULTROW{Low}\\
	\hline
	\MULTROW{Loss of GPS during landing maneuvers}[*][3] & Aircraft will estimate position, attitude and altitude using IMU, ultrasonic sensor and accelerometer, respectively & \MULTROW{Low}[*][3]\\
	\hline
	\MULTROW{Harm to personnel when arming aircraft} & Aircraft will wait for one minute before activating motors after arming & \MULTROW{Low}\\
	\hline
\end{tabularx} 
\caption{Risk Management - Autonomous Takeoff and Landing}
\end{table}

GPS failure in all modes, hovering without GPS

\begin{table}[!ht]
	\label{tab:management-inflight}
	\centering
	\begin{tabularx}{\textwidth}{|Y|Y|c|}
		\hline
		\MULTROW{\textbf{Risk}} & \MULTROW{\textbf{Control Measures}} & \MULTROW{\textbf{Residual Risk Rating}}[70pt]\\
		& & \\
		\hline
		Loss of aircraft control & \MULTROW{Autopilot implements failsafe termination} & \MULTROW{Low} \\
		(Autopilot failure or lock-up) & & \\
		\hline
		Loss of aircraft control (Propeller or power loss) & Failsafe is commanded by Raspberry Pi upon erratic/uncontrolled behaviour & \MULTROW{Low} \\		
		\hline
		Loss of GPS link only to GCS & Likely & Low \\
		\hline
		Loss of telemetry/radio link only to GCS & Likely & Low \\
		\hline
		Loss of both telemetry and GPS to GCS & Possible & Low \\
		\hline
		GCS failure & \MULTROW{Possible} & \MULTROW{Low}\\
		(aircraft loses communication with GCS) & & \\
		\hline
		Geofence breach & Likely & Medium \\
		\hline
	\end{tabularx} 
	\caption{Risk Assessment - In-flight Hazards}
\end{table}

\subsection{Battery Management}
What are we doing to ensure we have enough power to stay in the sky?